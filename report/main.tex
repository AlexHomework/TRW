\documentclass{article}
\usepackage[russian]{babel}
\usepackage[utf8]{inputenc}
\usepackage{graphics}
\usepackage{epsfig}
\usepackage{graphicx}
\usepackage{caption}
\usepackage{subcaption}
\usepackage{color}
\usepackage{hyperref}

\title
    {Решение задачи MAP для марковской сети типа решётка}
\author
    {Новиков~А.\,В.\\
    МГУ, ВМиК, каф. ММП}

\begin{document}
\maketitle

\pagebreak

\section{Постановка задачи}
Марковские сети применяются практически повсюду.
После обучение сети нужно делать её использовать,
то есть решать задачу максимизации апостериорной вероятности.
Для большинства реальных задач эта задача NP-сложная.\\
В данной статье проводится обзор существующих
state-of-the-art подходов к этой задаче.
\pagebreak

\section{Методы}
Мы остановимся на подклассе алгоритмов использующих
двойственное разложение. В них задача минимазации
исходной энергии сводится к задаче максимизации
двойственной энергии (это всегда строго выпулая функция).
Алгоритмы этой группы отличает конкретный метод максимизации.
\begin{itemize}
\item Покоординатный спуск~\cite{Coordinatewise}
\item Субградиентный спуск~\cite{Subgradient}
\item Bundle methods~\cite{Bundle}
\item <<Полная декомпозиция>>
\end{itemize}
\pagebreak


\begin{thebibliography}{1}
\bibitem{Coordinatewise}
    {Sontag~D.,  Jaakkola~T.}
    {New outer bounds on the marginal polytope}~//
    NIPS,~2008.
\bibitem{Subgradient}
    {Komodakis~N., Paragios~N., Tziritas~G.}
    {MRF energy minimization and beyond via dual decomposition}~//
    Pattern Analysis and Machine Intelligence, IEEE Transactions on, 2011.~--- С.\,531--552.
\bibitem{Bundle}
    {Kappes~J.\,H.}
    {A Bundle Approach To Efficient MAP-Inference by Lagrangian Relaxation}~//
    Computer Vision and Pattern Recognition (CVPR), IEEE Conference 2012.~--- С.\,1688--1695.
\end{thebibliography}
\end{document}
